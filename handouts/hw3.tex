\documentclass{exam}
\pagestyle{head}
\firstpageheader{Math 7670 -- Toric Varieties -- Suggested Problems \# 3 -- Spring 2019}{}{}

\usepackage{amsmath,amsthm,amssymb}
 
\newcommand{\N}{\mathbb{N}}
\newcommand{\Z}{\mathbb{Z}}
\newcommand{\rr}{\mathbb{R}}
\newcommand{\cc}{\mathbb{C}}
\newcommand{\zz}{\mathbb{Z}}
\newcommand{\ff}{\mathbb{F}}
 
\newenvironment{theorem}[2][Theorem]{\begin{trivlist}
\item[\hskip \labelsep {\bfseries #1}\hskip \labelsep {\bfseries #2.}]}{\end{trivlist}}
\newenvironment{lemma}[2][Lemma]{\begin{trivlist}
\item[\hskip \labelsep {\bfseries #1}\hskip \labelsep {\bfseries #2.}]}{\end{trivlist}}
\newenvironment{exercise}[2][Exercise]{\begin{trivlist}
\item[\hskip \labelsep {\bfseries #1}\hskip \labelsep {\bfseries #2.}]}{\end{trivlist}}
\newenvironment{problem}[2][Problem]{\begin{trivlist}
\item[\hskip \labelsep {\bfseries #1}\hskip \labelsep {\bfseries #2.}]}{\end{trivlist}}
\newenvironment{question}[2][Question]{\begin{trivlist}
\item[\hskip \labelsep {\bfseries #1}\hskip \labelsep {\bfseries #2.}]}{\end{trivlist}}
\newenvironment{corollary}[2][Corollary]{\begin{trivlist}
\item[\hskip \labelsep {\bfseries #1}\hskip \labelsep {\bfseries #2.}]}{\end{trivlist}}

\begin{document}

We will discuss these problems in class on February 28.
This set of questions involves taking what we have done for polyhedral cones, and obtaining
similar results for polytopes.

\bigskip
A {\bf polytope} $P$ in $V = \rr^n$ is the convex hull $P = conv(v_1, \ldots, v_m)$ of a finite number of points: i.e. there are
points $v_1, \ldots, v_m \in V$, such that
\[ P = \{ t_1 v_1 + \ldots + t_m v_m \mid t_i \ge 0, \sum_{i=1}^m t_i = 1 \} \subset V. \]

\medskip
A {\bf polyhedron} $P \subset V = \rr^n$ is the intersection of finitely many affine halfspaces:
i.e., there exists an $m \times n$ matrix $A$ and a vector $b \in \rr^m$ such that
  \[ P = \{ x \in V \mid  Ax \le b \} \subset V. \]

\medskip  
The {\bf cone $C(P)$ over $P$} is the set in $\rr \oplus V = \rr^{n+1}$:
  \[ C(P) = \{ r \begin{pmatrix}1 \\ x \end{pmatrix} \mid x \in P, r \ge 0 \}. \]
  We can prove many of the statements below by using our knowledge of cones and the relationship
  between $P$ and $C(P)$.

  \medskip  
  An {\bf affine hyperplane} in $V$ is a subset of the form
  \[ H_{u,b} := \{ x \in V \mid \langle u, x \rangle = b \}, \]
  where $b \in \rr$ and $u \ne 0 \in V^*$. We denote by
  \[ H^-_{u,b} := \{ x \in V \mid \langle u, x \rangle \le b \}, \]
  the negative half space determined by $u$ and $b$.
  $H_{u,b}$ is a {\bf supporting hyperplane} of a polytope or polyhedron $P$ if
  (i) $P$ is contained in $H^-_{u,b}$, and
  (ii) $P \cap H_{u,b} \ne \emptyset$.

\medskip  
  A (proper) {\bf face} $F \subset P$ is a subset of the form $P \cap H_{u,b}$, where $H_{u,b}$ is a supporting
  affine hyperplane of $P$.  We also consider the empty set and $P$ itself to be faces of $P$.

  \medskip
  Suppose that $P \subset V$ is a full dimensional polytope, which contains the origin in its interior.
  Define the {\bf polar dual} of $P$ to be
  \[ P^\circ := \{ u \in V^* \mid \langle x, u \rangle \ge -1 \} \subset V^* \]
  \medskip

  \medskip If $M = \zz^n$ is a lattice in $M_\rr = \rr^n$, then a polytope $P \subset M_\rr$ is called a
           {\bf lattice polytope} if its vertices lie in $M$.
           
  \begin{questions}
    \question Choose a polytope $P$ in $\rr^2$ or $\rr^3$.  Describe the above notions for your
    choice of $P$, and use your $P$ to exemplify the results in this problem set.
    
    \question Let $P \subset V$ be a subset.  Show that $P$ is a polytope if and only if
    $P$ is both bounded and a polyhedron.

    \question Show that a face of a polytope $P$ is also a polytope.
    
    \question Provide a definition of the dimension of a polytope (and therefore, for the dimension
    of any face of a polytope $P$).
    A vertex of $P$ is a face of dimension 0, an edge of $P$ is a face of dimension 1, and a facet
    is a face of dimension one less than the dimension of $P$.
    
    \question Show that every proper face of the polytope $P$ is contained in a facet of $P$.

    \question Show that every proper face of the polytope $P$ is the intersection of the facets of $P$
    that contain it.

    \question If $P$ is a full dimensional polytope containing the origin in its interior, then
    \begin{parts}
      \part $P^\circ$ is also a polytope.
      \part $P^{\circ\circ} = P$.
      \part If $F$ is a face of $P$, then
      \[ F^\circ := \{ u \in P^\circ \mid \langle x, u \rangle = -1, \mbox{for all $x \in F$} \} \]
      is a face of $P^\circ$ (notice that as a polytope, $F$ doesn't have a polar dual defined, so
      the notation isn't (too) ambiguous!).
      \part There is an order reversing bijection between the faces of $P$, and the faces of $P^\circ$
      given by sending $F$ to $F^\circ$,
      which satisfies $\dim F + \dim F^\circ = \dim P - 1$, for all faces $F$ of $P$.
    \end{parts}

    \question There are two ways to construct a fan from a lattice polytope, here is one of them:
    Let $P \subset N_\rr$ be a full dimensional lattice polytope with the origin in its interior.  Let
    $\Sigma$ be the set of all cones over faces of $P$.  Show that $\Sigma$ is a fan, and that
    this fan is complete.

    \question ({\bf Normal fan} of a polytope) Here is perhaps the most important method to obtain a fan from a lattice polytope:
    Let $P \subset M_\rr$ be a full dimensional lattice polytope.
    For each face $F \subset P$, define
    \[ \sigma_F := \{ u \in N_\rr \mid \langle x' - x, u \rangle \le 0, \mbox{ for all $x \in F$ and $x' \in P$} \}
    \]
    Let $\Sigma_P$ be the collection of all of the $\sigma_F$, for $F$ a face of $P$.

    \begin{parts}
      \part Show that $\Sigma_{kP} = \Sigma_P$, for any positive integer $k$.
      \part Show that if the origin is in the interior of $P$, then $\Sigma_P$ consists
      of the (negatives of the) cones over faces of $P^\circ$, as in the previous problem.
      \part Show that $\Sigma_P$ is a fan in $N_\rr$, and that this fan is complete.
      (There are a couple of ways to see this, but one way is to first note that
      for $k$ some positive integer, $kP$ can be translated so that the origin is an interior point).
    \end{parts}
    
  \end{questions}




\end{document}

  \question Let $\Sigma \subset N_\rr = \rr^2$ be a complete fan in the plane, and
  suppose that the corresponding toric variety is smooth.  Show that there is a fan $\Sigma'$
  such that (a) $\Sigma'$ has either 3 or 4 rays, and (b) $\Sigma$ is a refinement of $\Sigma'$.

  \question Let $\ff_2$ be the toric variety corresponding to the fan with rays $(1,0)$, $(0,1)$, $(-1,2)$ and $(0,-1)$.
  Find all of the $T$-invariant subvarieties of $\ff_2$.

