\documentclass[11pt]{article}
\usepackage{amssymb,amsmath}
\usepackage{array}
\usepackage{fancyheadings}

% see https://tex.stackexchange.com/questions/10684/vertical-space-in-lists
% for why we include this package.
\usepackage{paralist}


\parindent0pt
%\parskip1.3ex
\parskip=4pt %1.6ex

%\fontfamily{ptm}\selectfont
%\renewcommand{\rmdefault}{ptm}
%\renewcommand{\familydefault}{ptm}

\oddsidemargin-10mm 
\evensidemargin-10mm 
\textwidth183mm
\topmargin-25mm 
\textheight300mm

\setlength{\marginparwidth}{5mm}

\pagestyle{fancy}
%\headrulewidth0pt
\cfoot{}
\chead{}

\begin{document}


%\nopagenumbers
%\parskip=4pt
\hspace{1cm} {\bf Math 7670 -- Introduction to Toric Varieties -- Course Information -- Spring 2019}

\bigskip
\noindent
    {\bf Professor:}
    Michael Stillman, 503 Malott, mes15@cornell.edu

\smallskip
\noindent
    {\bf Lectures:}
    Tuesday and Thursdays 10:10 -- 11:25, MT 207

\smallskip
\noindent
    {\bf Website:}
    Piazza webpage: {\tt piazza.com/cornell/spring2019/math7670/home}

    Signup link: {\tt piazza.com/cornell/spring2019/math7670}

    If you do not have a netid, please contact me, and I can give you access to the site.
    All homework, announcements, notes, etc. will be posted on piazza.

    \medskip
\noindent
    {\bf Textbook(s):}
    Fulton, {\it Introduction to Toric Varieties}, 
    (optional) Cox, Little, Schenck, {\it Toric Varieties}.

\medskip
\noindent
    {\bf Prereqs:}
A first course in algebraic geometry (e.g. Shafarevich, Vol1,
or Hartshorne, chapter 1, or Hasset, or Cox-Little-O'Shea).  Some
knowledge of sheaves, divisors, and schemes, or concurrent enrollment
in Brian Hwang's algebraic geometry course, math 6670.


\medskip
\noindent
    {\bf Homework:}
I will hand out suggested homework problems, (which will NOT
be handed in), and we will discuss these in class.  Some problems will
be simple, making sure you understand the concepts, some will be
computational, and some will be more challenging, to allow you to
understand the material more deeply.  Finally, some might have you
explore some of the techniques that we will not have time to cover in
class.

\medskip
\noindent
    {\bf Projects:}
At the end of the semester, students will present short
talks on a subject in toric varieties of their interest.  I will have
lots of possible topics to suggest.

\medskip
\noindent
    {\bf Class notes:}
I will ask class participants to take turns taking notes, and
latex'ing them.  I will then post them on piazza.

\medskip
\noindent
    {\bf Course description:}
This course will be an introduction to toric varieties, assuming a
first course in algebraic geometry, and either basic knowledge of
sheaves, divisors, and schemes, or concurrent enrollment in Brian
Hwang's math 6670.

Toric varieties are wonderful and explicit examples of affine,
projective, and even more general varieties.  They provide an
extremely useful link of algebraic geometry with combinatorics.  Toric
varieties appear in many places, including in physics.

A tentative list of topics to be covered:

\begin{compactitem}
\item Polyhedral cones and polytopes.
\item Affine toric variety corresponding to a cone.
\item Polyhedral Fans.
\item Projective toric varieties, non-affine toric variety corresponding to a fan.
\item The torus action.
\item Singularities, smoothness, resolution of singularities
\item Divisors, properties of divisors (nef, big, ample, very ample),
    cones: nef cone, effective cone.
\item The homogeneous coordinate ring of a toric variety.
\item Line bundles
\item Cohomology (topological, of line bundles, of more general vector bundles)
\item Intersection theory
\end{compactitem}
Throughout, we will see many examples, and we will use the
NormalToricVarieties, Polyhedra, and FourierMotzkin packages in
Macaulay2 to explore examples.

A note about the relationship between this course and Brian Hwang's
course on algebraic geometry: We will coordinate our courses, so that
after Brian introduces a concept, we will examine that concept in our
course.  For instance, once he defines a scheme, we will see how to
glue affine toric varieties to produce more general toric varieties.

\medskip
\noindent
{\bf Grading:}
\noindent
S/U only. 



\end{document}

% Local Variables:
% compile-command: "pdftex blurb.tex && open blurb.pdf"
% End:
